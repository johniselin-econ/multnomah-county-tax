% Table 2: Migration Rates for Multnomah and Neighboring Counties
% Generated by 02_descriptives.do
% Requires: \usepackage{booktabs, threeparttable}
\begin{table}[htbp]
\centering
\begin{threeparttable}
\caption{Migration Rates for Multnomah County and Neighboring Counties}
\label{tab:migration_rates}
\small
\resizebox{\textwidth}{!}{%
\begin{tabular}{lccccc}
\toprule
 & \multicolumn{2}{c}{In-Migration Rate} & \multicolumn{2}{c}{Out-Migration Rate} & Change in \\
\cmidrule(lr){2-3} \cmidrule(lr){4-5}
County & 2018--19 & 2021--22 & 2018--19 & 2021--22 & Net Rate \\
\midrule
Multnomah & 8.76 & 8.50 & 8.50 & 9.67 & -1.44 \\
\addlinespace
\textit{Neighboring OR counties} & & & & & \\
\quad Washington & 8.33 & 8.10 & 8.01 & 8.49 & -0.71 \\
\quad Clackamas & 8.79 & 8.80 & 8.15 & 8.66 & -0.50 \\
\quad Marion & 6.74 & 6.88 & 6.08 & 6.77 & -0.55 \\
\quad Yamhill & 7.85 & 8.09 & 7.12 & 7.61 & -0.24 \\
\quad Columbia & 7.82 & 8.05 & 6.59 & 7.22 & -0.41 \\
\addlinespace
\textit{Neighboring WA counties} & & & & & \\
\quad Clark & 7.47 & 7.88 & 6.01 & 6.59 & -0.18 \\
\quad Skamania & 9.59 & 10.22 & 7.87 & 8.00 & 0.51 \\
\addlinespace
All other OR counties & 7.71 & 7.72 & 6.65 & 7.18 & -0.53 \\
All other WA counties & 7.35 & 7.35 & 6.69 & 7.33 & -0.64 \\
\bottomrule
\end{tabular}%
}
\begin{tablenotes}[flushleft]
\small
\item \textit{Notes:} Rates are domestic migration flows as a percentage of the base filing population (non-movers plus all movers) for each county-period. Pre-period averages tax years 2018--2019; post-period averages 2021--2022. Change in net rate is the shift in net in-migration rate (in-rate minus out-rate) between periods, in percentage points. Source: IRS Statistics of Income.
\end{tablenotes}
\end{threeparttable}
\end{table}
